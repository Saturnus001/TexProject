\subsection{数据结构——哈希表}\label{subsec:adt}
{{哈希表(Hash table,也叫哈希映射),是根据键(key)直接访问在内存存储位置的数据结构。
也就是说,它通过计算一个关于键值的函数,将所需查询的数据映射到表中一个位置来访问记录,
这加快了查找速度。这个映射函数称做散列函数,存放记录的数组称做散列表。}}

{{由于散列(hashing)是一种用于以常数平均时间执行插入、删除、查找的技术。
那些需要元素间任何排序信息的操作将不会得到有效支持。因此诸如FindMin、FindMax等操作都是哈希表不支持的。}}

{{在本实验中,哈希表采用开散列的方法实现。处理冲突时把散列到同一槽中的所有元素都放在一个链表中。
在查找时,先计算元素的哈希值,然后遍历该链表查找元素。通过拉链法(即链地址法)解决冲突有以下优点:}}

{{(1)拉链法处理冲突简单,且无堆积现象,即非同义词决不会发生冲突,因此平均查找长度较短;}}

{{(2)由于拉链法中各链表上的结点空间是动态申请的,故它更适合于造表前无法确定表长的情况;}}

{{(3)当结点较大时,拉链法中增加的指针域可忽略不计,因此节省空间;}}

{{(4)在用拉链法构造的散列表中,删除结点的操作易于实现。只要简单地删去链表上相应的结点即可。}}

\subsection{主程序流程及其模块调用关系}\label{subsec:relate}
\begin{tikzpicture}
	\node[draw, rounded corners](system){system};
	\node[draw, rounded corners,below=5pt of system](start){start};
	\node[draw, below=of start](CH){CreateHash//创建哈希表};
	\node[draw, below=of CH](V){CreateEigenvector};
	\node[draw, below=of V](S){SAsses//计算相似度};
	\node[draw, below=of S](D){DAsses//计算几何距离};
	\node[draw, rounded corners, below=20pt of D](end){End};

	\draw[->] (start)  -- (CH);
	\draw[->] (CH) -- (V);
	\draw[->] (V) -- (S);
	\draw[->] (S) -- (D);
	\draw[->] (D) -- (end);


	\node[draw, rounded corners,right=120pt of system](CV){CreateEigenvector};
	\node[draw, rounded corners,below=5pt of CV](start1){start};
	\node[draw, ellipse,below=of start1](L){While};
	\node[left =50pt of L](temp){};
	\node[draw, diamond, aspect=2, below=of L](eof){!fin.eof()};
	\node[draw, below=of eof](RW){ReadWord//读取单词};
	\node[draw, below=of RW](SH){SearchHash//查找表};
	\node[left =23pt of SH](temp1){};
	\node[draw, right=45pt of eof](EV){Eigenvector//得到特征向量};
	\node[draw, rounded corners, below=20pt of EV](end1){End};

	\draw[->] (start1)  -- (L);
	\draw[->] (L)  -- (eof);
	\draw[->] (eof) -- node[left]  {Yes} (RW);
	\draw[->] (RW)  -- (SH);
	\draw[->] (SH)  --(temp|-SH)--(temp)--(L);
	\draw[->] (SH)  --(temp1)--(temp1|-L)--(L);
	\draw[->] (eof) -- node[above] {No}  (EV);
	\draw[->] (EV) -- (end1);

\end{tikzpicture}
