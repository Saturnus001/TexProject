\subsection{问题分析与体会}\label{subsec:analysis}
{{本项目工程主要分为两个部分。
第一部分是哈希表的创建和查找,第二部分是根据查找结果得到向量以及对向量的数学计算等处理。}}

{在这两部分中,文件的读取和写入是必不可少的,因此,对文件的读写操作是调试的重点。
在本实验中,根据给定的文件,选用适当的函数读取文件,并且注意想要获取的内容间分隔符的处理,
是第一个难点,也是调试的重点。此外,文件读取状态的判断也是一个重点。}

{{在正确地读取文件内容后,如何处理向量也是实验重点。
在一开始,只是简单地将每个读取的向量的处理步骤转换成代码,造成了代码臃肿,复用性差,并且难以调试的问题。
通过对代码的重构和优化,将向量的处理步骤封装成通用的函数,使得代码的复用性大大提高,
并且易于调试,在实验过程中数学运算的错误也更易发现。}}

{{在本次实验中,注释发挥了重要的作用。通过对一些细节操作的注释,大幅加快了调试的速度。在遇到问题时能够很快找到解决方案。
注释也大幅提高了代码可读性,在优化代码时,注释作为参考,指明了数据初始化的状态,重要操作的目的,方便了后期的维护与修改。}}

{{通过这次实验,我锻炼了自己处理多个文件的能力,
认识了注释等好的编程习惯的重要性,提升了自己对非单一文件的项目工程的函数编写封装思路的理解。
它不仅加深了我对哈希表的认识,了解了哈希表在查重方面的应用,还提高了自己代码的编写能力,能够更好更快地写出容易理解且便于调试的代码。}}

\subsection{时空复杂度分析}\label{subsec:analysis2}
\subsubsection{时间复杂度}
{{在初始化部分,由于哈希表通过散列函数查找元素的性质,哈希表的创建和查找操作的时间复杂度均为$O(1)$。
在特征向量的运算部分,由于每次运算需要遍历向量的每个元素,时间复杂度为$O(n)$。}}

\subsubsection{空间复杂度}
{{在本实验中,哈希表的空间复杂度为$O(n)$}}

\subsubsection{平均查找长度}
{{由计算得,成功时平均查找长度:}}
\begin{equation}
	ASL=\frac{1\times13+2\times2+3\times1+4\times1}{17}\approx1.4<2\label{eq:equation2}
\end{equation}\\
{{而失败时,平均查找长度:}}
\begin{equation}
	ASL=\frac{1\times11+4\times1+2\times1}{43}\approx0.40<2\label{eq:equation3}
\end{equation}\\
{{符合平均查找长度小于2的要求。}}
