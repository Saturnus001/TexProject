\documentclass[a4paper]{article}
\usepackage[margin=1in]{geometry}
\usepackage[fontset=founder]{ctex}
\usepackage{anyfontsize}

\title{\textbf{认知与体会}}
\author{组名: 设计鬼才 \\ \ 姚苏航\ \ PB22061220}
\date{}

\begin{document}

	\maketitle

    {
        在我们小组的设计创新思维课题中, 以改善大学生宿舍中存在的生活问题为出发点, 
        通过深入的调研与用户交流, 选定噪声问题作为切入点, 努力结合实际寻找思路, 
        提供更切实可行的解决方案。

        在调研过程中, 我也对于创新的过程有了一些认知与体会。下面, 我将从七个方面谈谈我的收获。
    }

	\section{发散思维}

    {
        在进行设计创新思维的课题研究时, 我们小组采用了发散思维的方法, 
        从问题空间和调研范围出发, 尽可能地拓展了我们的思考边界。
        通过发散思维, 我们发现宿舍噪音问题远不止于传统的“室友噪音”问题。

        实际中, 大学生在寝室可能面临到来自走廊、隔壁、甚至是窗外的各种噪音。
        此外, 我们意识到宿舍生活中存在的噪音问题不仅仅局限于单一的来源, 而是涉及到灯光、噪音、作息等多个方面。
        无论是早八同学的闹钟声, 还是晚睡同学洗漱的声音, 甚至包括宿舍的灯光, 都会给同学们的寝室生活带来困扰。
        
        这种拓展的思考范围让我们认识到, 解决方案不仅需要面对同寝室的室友关系, 
        还需要考虑整个宿舍楼层的生活环境, 这有助于我们为大学生提供更为全面和实际的解决方案。
    }

    \section{以人为本}

    {
        我们始终将用户放在设计的核心位置, 着眼于大学生这一特定的用户群体。
        在与大学生的深入交流中, 我们发现每个人对噪音的敏感度和对解决方案的期望各不相同。
        有的同学希望解决方案能够在私人空间内改善噪音问题。
        而另一些同学可能更希望通过共同的规则或设备, 实现宿舍大环境的安静舒适。
        这些反馈使得我们明确了用户的差异性需求, 从而更精准地制定方案。

        通过调研, 我们深入了解了大学生在宿舍中的生活, 关注到他们在日常生活中所面临的噪音问题。
        这让我们对用户需求有了更深刻的理解。我们发现大学生对于宿舍生活中的噪音问题有很高的敏感度。
        他们对良好休息环境表现出了迫切需求, 不少同学认为自己的学习生活受到了噪声的困扰。

        这种以人为本的思考方式使我们更深入地理解了用户需求, 找到了核心痛点, 感受到了他们对解决方案的期待。        
    }

    \section{实质交流}

    {
        与用户的交流是我们调研过程中的关键一环。
        我们采用了多种方式, 包括线上问卷、面对面访谈以及小组内部的讨论。
        这种实质性的交流使我们更好地捕捉到用户的真实反馈, 
        了解到他们在宿舍生活中所面临的各种问题。
        与用户的深入交流不仅让我们更好地理解问题的本质, 
        也激发了我们设计解决方案的动力。

        通过问卷调查, 我们深入了解到大学生对于宿舍噪音问题的真实感受。
        有的同学表示在考试前特别需要一个相对安静的环境, 
        而有的同学则认为在集体活动中, 适当的噪音反而是生活的一部分。
        这些不同的需求反馈让我们清晰了目标人群。

        通过一对一访谈的方式, 我们了解到一些同学在宿舍中采取的各种尝试性的解决方案, 
        例如佩戴耳塞、调整作息时间等。

        但是, 调查交流的过程并非一帆风顺。
        在调研过程中, 我们发现有些同学缺乏兴趣, 认为噪音是寝室生活的一部分, 不需要解决。
        通过深入的调研与用户交流, 我们理解了他们的心态, 也发现了一些之前被忽视的问题。
        这种实质性的交流帮助我们认清目标群体, 能够更有针对性地提出解决方案。  
    }

    \section{原型风暴}

    {
        在讨论创新方向时, 我们采用了设计原型的方式。原型可以简明地阐述产品设计理念并进行验证。
        原型风暴过程中, 小组成员积极参与, 共同讨论并提出改进意见, 
        为后期产品的设计提供了大量宝贵素材。
    }

    \section{加速学习}

    {
        在调研和设计的过程中, 我们发现需要学习的知识不仅仅局限于设计理论和技术, 
        还包括了心理学、社会学等领域。
        因为了解大学生的心理状态和社交行为对于设计解决方案至关重要。
        已经学到的知识涉及到了社交心理学、声学设计等领域, 
        而我们计划继续学习关于智能设备技术的知识, 以更好地支持我们的设计。

        在这个过程中, 我们发现需要学习的知识涉及到室内设计、材料工程、智能调控技术等多个领域。
        为了更好地应对这些挑战, 小组成员积极主动地寻找学习资源, 参与相关课程和培训。
        我们已经学到了室内环境设计的基本原理、隔音材料的选择方法等。
        同时, 我们也计划继续学习智能调控技术, 以更好地支持我们的设计。

        在调研和解决方案设计的过程中, 我们认识到需要更深入地了解室内设计和智能调控技术。
        其中, 一些同学在学科知识上表现得更为积极, 他们主动参与了相关的课程和研讨会。
        然而, 一些技术性的问题仍然需要我们共同努力去学习。
        我们计划利用在线资源和专业导师的帮助, 深入研究这些领域, 以更好地支持我们的设计。
    }

    \section{不畏艰难}

    {
        在调研过程中, 我们遇到了一些意料之外的问题, 
        例如个别同学对噪音问题的淡漠态度, 以及影响因素的复杂性等等。
        加上团队成员之前并不熟悉, 彼此缺乏了解。
        这给我们带来了一些心理上的压力, 也拖慢了我们小组课题的进度。
        通过团队成员之间的鼓励和互相支持, 我们加强了沟通交流, 明确了每个人的职责和任务, 确保项目能够顺利进行。
        最终克服了困难。
    }

    \section{团队合作}

    {
        目前, 我们小组的团队合作相当融洽。
        虽然小组成员之间偶尔意见相左, 但都能通过交流沟通达成共识。
        每个成员在团队中扮演不同的角色, 各司其职, 共同推动课题的进展。
        然而, 我们也发现在沟通方面还存在一些潜在的问题, 有时候信息传递不够及时, 导致一些工作出现偏差。
        为了进一步提高团队合作效率, 我们计划增加团队例会的频率, 加强信息共享, 
        确保每个成员都能够及时了解目前选题的最新进展。
    }

    {
        总体而言, 这个设计创新思维的过程对我们小组而言是一次宝贵的经验。我相信我们小组能够不断找出不足、提升自己, 更好地改善大学生在宿舍中的生活环境, 
        提供一个创新且符合实际需求的解决方案。

    }

\end{document}