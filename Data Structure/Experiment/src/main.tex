%! Author = admin
%! Date = 2023/12/17

% Preamble
\documentclass[a4paper]{article}

% Packages
\usepackage[margin=1in]{geometry}
\usepackage[fontset=founder]{ctex}
\usepackage{anyfontsize}
\usepackage{graphicx}
\usepackage{amsmath}
\usepackage{amssymb}
\usepackage{mathabx}
\usepackage{multirow}
\usepackage{subfig}

\usepackage{hyperref}

\graphicspath{{../figures/}}

\title{\textbf{数据结构实验报告}}
\author{姚苏航\qquad PB22061220}
\date{}

% Document
\begin{document}
    \maketitle


    \section{问题描述}\label{sec:des}

    \subsection{实验题目}\label{subsec:q}
    {{利用哈希表统计两源程序的相似性。}}

    \subsection{基本要求}\label{subsec:req}
    {{对于两个C语言的源程序清单,
    用哈希表的方法分别统计两程序中使用C语言关键字的情况,
    并最终按定量的计算结果,得出两份源程序的相似性。}}

    \subsection{测试数据}\label{subsec:test}
    {{事先给出的file文件夹,包含关键词表和三份源程序文件,程序之间有相近的和差别大的。
    文件内容详见附录\ref{sec:appendix2}。}}


    \section{需求分析}\label{sec:need}

    \noindent{1.}

    \noindent{2.}

    \noindent{3.}

    \noindent{4.}


    \section{概要设计}\label{sec:design1}

    \subsection{所用到得数据结构及其ADT}\label{subsec:adt}

    \subsection{主程序流程及其模块调用关系}\label{subsec:relate}

    \subsection{核心模块的算法伪码}\label{subsec:code}


    \section{详细设计}\label{sec:design2}

    \subsection{实现概要设计中的数据结构ADT}\label{subsec:adt2}

    \subsection{实现每个操作的伪码,重点语句加注释}\label{subsec:explain}

    \subsection{主程序和其他模块的伪码}\label{subsec:code2}


    \section{调试分析}\label{sec:debug}

    \subsection{问题分析与体会}\label{subsec:analysis}

    \subsection{时空复杂度分析}\label{subsec:analysis2}


    \section{使用说明}\label{sec:instrut}
    {{用户将事先准备好的关键词表(keyword.txt)和需要统计相似性的程序放入file文件夹中,
    运行时程序将通过关键词表建立哈希表,并通过查找哈希表构建两程序的向量,
    最后通过判别函数计算两程序相似性和向量的几何距离。}}

    {{在本项目中,使用事先给出的测试数据,
    准备三个编译和运行都无误的C程序,程序之间有相近的和差别大的,
    通过similar.c和different.c两个程序与main.c进行比较,
    可以直观展现出比较的效果。}}


    \section{测试结果}\label{sec:result}

    \subsection{输入数据}\label{subsec:in}
    {{输入数据从给出的测试文件中读取,
    读取keyword.txt文件生成哈希表,
    再分别读取main.c,similar.c,different.c并进行比较}}

    \subsection{输出数据}\label{subsec:out}
    \noindent{输出结果显示在终端,内容如下:}

    \noindent{X\_(../file/similar.c):}

    \noindent{0 1 1 4 0 1 0 3 3 1 2 1 2 0 1 1 4}

    \noindent{X\_(../file/different.c):}

    \noindent{0 1 0 1 0 0 1 2 1 0 3 0 0 0 1 3 2}

    \noindent{X\_(../file/main.c):}

    \noindent{0 1 1 3 0 1 0 3 2 1 2 1 2 0 1 1 4}

    \noindent{ }

    \noindent{S\_(Sim\&Main):0.988174}

    \noindent{D\_(Sim\&Main):1.41421}

    \noindent{ }

    \noindent{S\_(Dif\&Main):0.740121}

    \noindent{D\_(Dif\&Main):4.89898}


    \appendix


    \section{实验源代码文件}\label{sec:appendix1}
    {{为方便查看,附录中的链接文件均为txt格式}}

    \href{../exp6/define.h.txt}{\underline{define.h}}

    \href{../exp6/main.cpp.txt}{\underline{main.cpp}}

    \href{../exp6/OpenHashing.h.txt}{\underline{OpenHashing.h}}

    \href{../exp6/OpenHashing.cpp.txt}{\underline{OpenHashing.cpp}}

    \href{../exp6/system.h.txt}{\underline{system.h}}

    \href{../exp6/system.cpp.txt}{\underline{system.cpp}}

    \href{../exp6/SimAsses.h.txt}{\underline{SimAsses.h}}

    \href{../exp6/SimAsses.cpp.txt}{\underline{SimAsses.cpp}}


    \section{实验用测试文件}\label{sec:appendix2}
    \href{../exp6/file/main.c.txt}{\underline{main.c}}

    \href{../exp6/file/different.c.txt}{\underline{different.c}}

    \href{../exp6/file/similar.c.txt}{\underline{similar.c}}

    \href{../exp6/file/keyword.txt}{\underline{keyword.txt}}


\end{document}
